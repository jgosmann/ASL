\documentclass[11pt,a4paper]{scrreprt}

%%%%%% Packages %%%%%
\usepackage[utf8]{inputenc}
\usepackage[T1]{fontenc}
\usepackage{ae,aecompl}
%\usepackage{scrpage2}
%\usepackage{amssymb}
%\usepackage{amsmath}
%\usepackage[flushmargin,ragged]{footmisc}
% \usepackage{graphicx}
%\usepackage{ifthen}
% \usepackage{lastpage}
%\usepackage{ngerman}
\usepackage{enumerate}
% \usepackage{hhline}
\usepackage{listings}
%\usepackage{xr-hyper}
\usepackage{hyperref}
%\usepackage{units}
% \usepackage{xcolor}

%%%%% Eigene Format-Kommandos %%%%%

%%%%% Externe Dokumente %%%%%

%%%%% Silbentrennung %%%%%

%%%%% Weiteres %%%%%

\title{Guide to the Annotated Shader Language}
\author{Jan Gosmann \and Denis John \and Markus Kastrop \and Andreas Mrugalla}
\hypersetup{
  pdftitle={Guide to the Annotated Shader Language},
  pdfauthor={Jan Gosmann, Denis John, Markus Kastrop, Andreas Mrugalla},
  pdfkeywords={OpenGL, Shader, ASL}
}

\begin{document}
\lstdefinelanguage[opengl]{Shader}[ANSI]{C}{morekeywords={in,sampler2D,uniform,varying,vec3}}
\lstdefinelanguage[ASL]{Shader}[opengl]{Shader}{morekeywords={Name,Description,Default,Range,Control},moredelim=*[s][\itshape]{/**}{*/},moredelim=[is][\itshape]{|}{|}}
\lstset{basicstyle=\fontfamily{pcr}\selectfont\footnotesize,keywordstyle=\bfseries,language=[ASL]Shader}
\maketitle
\tableofcontents

\chapter{Introduction}
In the summer term of 2011 at the Bielefeld University we, the authors of this 
guide, developed an image manipulation program which implemented its algorithms 
in shaders running on the GPU. This project was part of the Computer Graphics 
module.

We aimed to make this image manipulation program flexible and extensible.  
Therefore we decided to save the source code of the shaders in their own files 
which will be loaded and compiled and runtime. This makes it easy to add new 
algorithms: just write a shader and store it in the right directory.

However, it would be nice if one could control the various parameters of the 
algorithms, but a normal shader program does not provide the information what 
the purpose and valid range of a parameter is. This is the point were the 
Annotated Shader Language comes into play.

The Annotated Shader Language, or short ASL, allows it to add information like 
description and parameter range of a parameter to the program. Because all ASL 
elements are written in comments the shader program can still be compiled 
without using an ASL shader compiler.

Because developed ASL only as part of a rather small project there is still room 
for many enhancements. For example, at the moment it is only possible to 
annotate uniforms and whole files, but no per vertex parameters. We had no need 
for this function as we only used fragment shaders which do not have per vertex 
parameters.


\chapter{A First Look}
This chapter shows a simple example of an ASL annotated shader program to get 
a first impression what the language looks like:

\begin{lstlisting}
uniform sampler2D tex;

/**
 * Name: |Threshold|
 * Description: |This gives the threshold for coloring bright pixels. All pixels
 *     that have a color vector with a length greater than this value will be
 *     colored.|
 * Default: |1.33|
 * Range: |0, 1.44|
 */
uniform float threshold;

/**
 * Name: |Highlight Color|
 * Description: |Color used for highlighting the bright pixels.|
 * Default: |(0, 0, 0)|
 * Range: |percent|
 * Control: |color, vector|
 */
uniform vec3 color;

/**
 * Name: |High-Key|
 * Description: |Highlights bright areas.|
 */
void main() {
    gl_FragColor = gl_Color
        * texture2D(tex, gl_TexCoord[0].xy);
    if (length(gl_FragColor.rgb) > threshold) {
        gl_FragColor.rgb = color;
    }
}
\end{lstlisting}

This shader colors all bright pixels by computing the vector length of the 
color's RGB vector. As you can see the uniforms \lstinline$threshold$ and 
\lstinline$color$ are annotated with additional information such as the name, 
a description, default value and allowed parameter range. This allows to the 
program using this shader to provide corresponding user interface elements.

The uniform \lstinline$tex$ is expected to be passed to all shaders by the 
program using the shader and is not considered as a parameter the user can 
directly change per shader. Therefore it has no annotation.

Not only the uniforms, but also the \lstinline$main()$ function is annotated 
with some general information about the shader.

\chapter{Annotation Comments}

\section{General Shader Annotation}

\section{Uniform Annotation}

\chapter{Additional Features}

\section{MAIN Macro}

\end{document}
