\documentclass[11pt,a4paper]{scrreprt}

%%%%%% Packages %%%%%
\usepackage[utf8]{inputenc}
\usepackage[T1]{fontenc}
\usepackage{ae,aecompl}
%\usepackage{scrpage2}
%\usepackage{amssymb}
%\usepackage{amsmath}
%\usepackage[flushmargin,ragged]{footmisc}
% \usepackage{graphicx}
%\usepackage{ifthen}
% \usepackage{lastpage}
%\usepackage{ngerman}
\usepackage{enumerate}
% \usepackage{hhline}
\usepackage{listings}
%\usepackage{xr-hyper}
\usepackage{hyperref}
%\usepackage{units}
% \usepackage{xcolor}

%%%%% Eigene Format-Kommandos %%%%%
\newcommand{\key}[1]{\subsection{Key: #1}}

%%%%% Externe Dokumente %%%%%

%%%%% Silbentrennung %%%%%

%%%%% Weiteres %%%%%

\title{Guide to the Annotated Shading Language}
\author{Jan Gosmann \and Denis John \and Markus Kastrop}
\hypersetup{
  pdftitle={Guide to the Annotated Shading Language},
  pdfauthor={Jan Gosmann, Denis John, Markus Kastrop},
  pdfkeywords={OpenGL, Shader, ASL}
}

\begin{document}
\lstdefinelanguage[opengl]{Shader}[ANSI]{C}{morekeywords={in,sampler2D,uniform,varying,vec3}}
\lstdefinelanguage[ASL]{Shader}[opengl]{Shader}{morekeywords={Name,Control,Description,Default,Depends,Range,ShaderName,ShaderDescription},moredelim=*[s][\itshape]{/**}{*/},moredelim=[is][\itshape]{|}{|}}
\lstset{basicstyle=\fontfamily{pcr}\selectfont\footnotesize,keywordstyle=\bfseries,language=[ASL]Shader}
\maketitle
\tableofcontents

\chapter{Introduction}
In the summer term of 2011 at the Bielefeld University we, the authors of this 
guide, developed an image manipulation program which implemented its algorithms 
in shaders running on the GPU. This project was part of the Computer Graphics 
module.

We aimed to make this image manipulation program flexible and extensible.  
Therefore, we decided to save the source code of the shaders in their own files 
which will be loaded and compiled at runtime. This makes it easy to add new 
algorithms: just write a shader and store it in the right directory.

However, it would be nice if one could control the various parameters of the 
algorithms with GUI controls. But a normal shader program does not provide the 
information what the purpose and valid range of a parameter is. This is the 
point were the Annotated Shading Language comes into play.

The Annotated Shading Language, or short ASL, allows to add information like 
description and parameter range of a parameter to the program. Because all ASL 
elements are written within comments, the shader program can still be compiled 
without using an ASL shader compiler.

Because ASL was developed only as part of a rather small project there is still 
room for many enhancements. For example, at the moment it is only possible to 
annotate uniforms and whole files, but not per vertex parameters. We had no need 
for this function as we only used fragment shaders which do not have per vertex 
parameters.


\chapter{A First Look}
This chapter shows a simple example of an ASL annotated shader program to get 
a first impression what the language looks like:

\begin{lstlisting}
/**
 * ShaderName: |High-Key|
 * ShaderDescription: |Highlights bright areas.|
 */

uniform sampler2D tex;

/**
 * Name: |Threshold|
 * Description: |This gives the threshold for coloring bright pixels. All
 *     pixels that have a color vector with a length greater than this value
 *     will be colored.|
 * Default: |1.33|
 * Range: |0, 1.44|
 */
uniform float threshold;

/**
 * Name: |Highlight Color|
 * Description: |Color used for highlighting the bright pixels.|
 * Default: |vec3(0, 0, 0)|
 * Range: |percent|
 * Control: |color, vector|
 */
uniform vec3 color;

void main() {
    gl_FragColor = gl_Color
        * texture2D(tex, gl_TexCoord[0].xy);
    if (length(gl_FragColor.rgb) > threshold) {
        gl_FragColor.rgb = color;
    }
}
\end{lstlisting}
This shader colors all bright pixels by computing the vector length of the 
color's RGB vector. As you can see the uniforms \lstinline$threshold$ and 
\lstinline$color$ are annotated with additional information such as the name, 
a description, default value and allowed parameter range. This allows to the 
program using this shader to provide corresponding user interface elements.

The uniform \lstinline$tex$ is expected to be passed to all shaders by the 
program using the shader and is not considered as a parameter the user can 
directly change per shader. Therefore it has no annotation.

Please note that the uniform \lstinline$tex$ has no annotation! The first 
annotation comment provides always general information about the program.

\chapter{Annotation Comments}
Annotations are added to the shader program with special comments starting with 
two asterisks:
\begin{lstlisting}
/** <annotation comment> */
\end{lstlisting}
The comment has to occur before the element it is annotating.

Annotation comments can span multiple lines:
\begin{lstlisting}
/**
 * <line 1>
 * <line 2>
 */
\end{lstlisting}
All whitespace and asterisks before and after the line will be discarded. This 
allows also boxlike annotation comments:
\begin{lstlisting}
/**************************
 * <annotation comment>   *
 * <another line>         *
 *************************/
\end{lstlisting}

Within an annotation comment you define the individual annotations by 
a key-value pattern. The key is the first word in the line followed by a colon 
and than the value attributed to that key:
\begin{lstlisting}
/**
 * key: value
 */
\end{lstlisting}
The keys are case-sensitive.

Sometimes it is useful to split the value into multiple lines, e.g. when writing 
long descriptions. To do this you only have to indent the lines after the first.  
All lines with a higher indentation than the key will be treated as 
a continuation of the value.
\begin{lstlisting}
/**
 * key: with a value
 *     that spans more
 *     than one line.
 * another-key: value
 */
\end{lstlisting}
Tabs will be considered as 8 spaces in this context. However, you should not mix 
tabulators and spaces.

\section{General Shader Annotation}
There are some annotations which are not used as annotation for a single element 
like a parameter, but for the whole shader program. These annotations have to 
occur in a special annotation comment which has to be at the start of the file.  
Only preprocessor directives and white spaces are allowed to precede it.

Please note that you have to include this general annotation comment in any ASL 
program to be treated as an ASL program. If you want to assign the default 
values to all keys you may just use \lstinline$/***/$ (the first two asterisks 
introduce the ASL Comment and the third asterisk and last slash close it 
immediately).

\key{ShaderName}
Gives the name of the shader program. Should be short. If this annotation is 
missing it defaults to the filename of the shader program.
\key{ShaderDescription}
A description of the shader program.

\key{Depends}
It might be useful to use the same function in different shaders. This can be 
done with the \lstinline$Depends$ key. You may view it as an include function 
even though it does not really include files, but compiles the files 
individually and links them together.

The \lstinline$Depends$ key takes a comma separated list of paths to other 
shader programs. At the moment the name of a dependency must not contain 
a comma. Moreover, each path may not use more than one line. Use always a slash 
`/' as directory separator.

The files ``included'' this way can specify their own dependencies. In the 
dependency list may even be duplicates which will only be linked once to the 
complete program.

The procedure by which the dependencies may be compiler specific. However, it is
required to look for files relative to the file specifying the dependencies 
first.

\subsection{Example}
\begin{lstlisting}
/**
 * ShaderName: Example Shader
 * ShaderDescription: This is just an example with a shader doing nothing.
 * Depends: some-other-shader.fs, and-another-shader.fs
 */
\end{lstlisting}

\section{Uniform Annotation}
To annotate a uniform parameter of a shader program write an annotation comment 
in front of it.

The data type of the parameter should be taken from the parameter declaration by 
the ASL compiler. Our current implementation does not support samplers, but only 
scalars, vectors and matrices as data types for annotated parameters. Moreover, 
qualifiers like the \lstinline$layout$ qualifier and arrays are not supported.

The allowed keys for annotations are described in the following subsections.

\key{Name}
Gives the name of the parameter. Should be short. This annotation exists because 
the variable names in programs are often not that clear without context. This 
annotation allows you to add a more clear name for the user interface. You also 
do not have to worry about characters not valid in variable names.

This annotation default to the identifier of the annotated parameter.

\key{Description}
A description what the annotated parameter does. Programs using ASL may present 
this description as tool tip or any other way that seems suitable to them.

\key{Default}
Gives the default value for the parameter. Specifying the value works the same 
as specifying a value in the standard shading language. However, it is not 
possible to use calculations or expressions like \lstinline$3 * 60$.

The ASL compiler tries to cast the type of the value to the type of the 
annotated uniform.

\key{Range}
This gives the valid parameter range. The basic syntax is the minimum and 
maximum value separated by a comma:
\begin{lstlisting}
Range: 0, 1
\end{lstlisting}
The allows the parameter to be between 0 and 1 whereby the bounds are included.

You can also specify the valid range for vectors and matrices. If you use the 
scalar form from above the range is applied to each element of the vector or 
matrix. To specify individual ranges for each element write one vector 
(respectively matrix) with the minimum values and one with the maximum values 
like this:
\begin{lstlisting}
Range: vec3(0, 0, -5), vec3(10, 10, 5)
\end{lstlisting}

If you want to use the smallest or highest possible value you do not have to 
type the numeric value. It is better to use the words \lstinline$min$ and 
\lstinline$max$:
\begin{lstlisting}
Range: min, max
\end{lstlisting}
This would be the equivalent to not specifying \lstinline$Range$ at all.

Instead of specifying the ranges manually there are some predefined ranges:
\begin{itemize}
    \item \verb$percent$ is the same as \lstinline$0, 1$
    \item \verb$byte$ is the same as \lstinline$0, 255$
    \item \verb$unsigned$ is the same as \lstinline$0, max$
    \item \verb$positive$ is the same as \lstinline$1, max$
\end{itemize}

\key{Control}
A vector may either represent a position or a color. In each case another 
control would be suitable in the user interface. With the \lstinline$Control$ 
key you can give the program using this shader a hint which control to use.

The value of \lstinline$Control$ key is a comma separated list. The program 
using the shader should use the first control it provides in the list. If the 
program provides none of the controls in the list it should fall back to its 
default control for the data type.

Each program may define it own controls it provides. However, there are some 
standard names for controls you should use; even though a program is not 
required to provide these.
\begin{itemize}
    \item \verb$default$ stands for the program's default control for the data 
        type of the parameter.
    \item \verb$color$ request the program to use a color picker as control.  
        Makes sense only for vectors.
    \item \verb$slider$ requests to use a slider as control. (In case of 
        a non-scalar this may be interpreted as per element.)
    \item \verb$spinbox$ requests to use a spin box as control. (In case of 
        a non-scalar this may be interpreted as per element.)
\end{itemize}

\subsection{Example}
\begin{lstlisting}
/**
 * Name: Highlight Color
 * Description: The color used to highlight the affected areas.
 * Default: vec3(1, 0, 0)
 * Range: percent
 * Control: color, default
 */
uniform vec3 highlightColor;
\end{lstlisting}
This is a possible annotation for a color value using three components (RGB).  
Each component is limited to the range from 0 to 1 and the program should use 
a color picker to control this parameter.

\chapter{Additional Features}
Besides annotating elements in the shader program for usage in the user 
interface there are some additional features which are documented in this 
chapter.

\section{ASL\_MAIN Macro}
ASL provides the \lstinline$ASL_MAIN$ macro. This will be set only for the 
top-level file being compiled and not for files being compiled because they 
occurred within a \lstinline$Depends$ key.

This allows you to write ASL shaders which can operate either as standalone ASL 
shader or provide functionality to other ASL shaders.

Here is an example:
\begin{lstlisting}
/**
 * ShaderName: Convoluter
 * ShaderDescription: Convolutes the a texture with a given matrix.
 */

#ifdef ASL_MAIN

/**
 * Name: Convolution Matrix
 */
uniform mat3 convolutionMatrix;

void main() {
    convolute(convolutionMatrix);
}

#endif

uniform sampler2D texture;

void convolute(mat3 convMat) {
    /* Convolution code here. */
}
\end{lstlisting}

\section{Standard Uniforms}
Not all uniform parameters of a shader should be directly user controlled and 
therefore passed by the program to the shader without providing a control in the 
user interface. In ASL you just do not add an annotation comment for those 
parameters which prevents the program from generating controls for it in the 
user interface.

However, the parameter has still to be set to the correct value and therefore 
the program and the shader have to use the same name for the parameter. To be 
able to use the same shader with different programs as long as the shader makes 
sense in the program's context all programs and shaders should use the naming 
conventions in this chapter.

\subsection{Fragment Shaders}
\begin{itemize}
    \item \lstinline$uniform sampler2D tex$: The texture used by the shader.
    \item \lstinline$uniform int texWidth$: The width of the texture.
    \item \lstinline$uniform int texHeight$: The height of the texture.
\end{itemize}

\appendix
\chapter{Limitations and Possible Future Enhancements}
Because ASL was developed as part of a rather small project it is far from 
complete and leaves much room for improvements. We focused on fragment shaders 
as these were the only shaders we used in the project. Therefore we did not 
considered any special needs which might arise when using ASL with vertex or 
geometry shaders. For example we do not provide any support for annotating 
``varying'' respective ``in'' parameters.

Moreover we had no use for parameters of the type sampler. Because of that our 
compiler is not able to create use of annotated parameters of that type.

It might also be nice to be able use expressions like \lstinline$3 * 42$ for 
specifying default, minimum and maximum value.

\end{document}
