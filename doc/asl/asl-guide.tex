\documentclass[11pt,a4paper]{scrreprt}

%%%%%% Packages %%%%%
\usepackage[utf8]{inputenc}
\usepackage[T1]{fontenc}
\usepackage{ae,aecompl}
%\usepackage{scrpage2}
%\usepackage{amssymb}
%\usepackage{amsmath}
%\usepackage[flushmargin,ragged]{footmisc}
% \usepackage{graphicx}
%\usepackage{ifthen}
% \usepackage{lastpage}
%\usepackage{ngerman}
\usepackage{enumerate}
% \usepackage{hhline}
\usepackage{listings}
%\usepackage{xr-hyper}
\usepackage{hyperref}
%\usepackage{units}
% \usepackage{xcolor}

%%%%% Eigene Format-Kommandos %%%%%
\newcommand{\key}[1]{\subsection{Key: #1}}

%%%%% Externe Dokumente %%%%%

%%%%% Silbentrennung %%%%%

%%%%% Weiteres %%%%%

\title{Guide to the Annotated Shader Language}
\author{Jan Gosmann \and Denis John \and Markus Kastrop \and Andreas Mrugalla}
\hypersetup{
  pdftitle={Guide to the Annotated Shader Language},
  pdfauthor={Jan Gosmann, Denis John, Markus Kastrop, Andreas Mrugalla},
  pdfkeywords={OpenGL, Shader, ASL}
}

\begin{document}
\lstdefinelanguage[opengl]{Shader}[ANSI]{C}{morekeywords={in,sampler2D,uniform,varying,vec3}}
\lstdefinelanguage[ASL]{Shader}[opengl]{Shader}{morekeywords={Name,Description,Default,Include,Range,Control},moredelim=*[s][\itshape]{/**}{*/},moredelim=[is][\itshape]{|}{|}}
\lstset{basicstyle=\fontfamily{pcr}\selectfont\footnotesize,keywordstyle=\bfseries,language=[ASL]Shader}
\maketitle
\tableofcontents

\chapter{Introduction}
In the summer term of 2011 at the Bielefeld University we, the authors of this 
guide, developed an image manipulation program which implemented its algorithms 
in shaders running on the GPU. This project was part of the Computer Graphics 
module.

We aimed to make this image manipulation program flexible and extensible.  
Therefore we decided to save the source code of the shaders in their own files 
which will be loaded and compiled and runtime. This makes it easy to add new 
algorithms: just write a shader and store it in the right directory.

However, it would be nice if one could control the various parameters of the 
algorithms, but a normal shader program does not provide the information what 
the purpose and valid range of a parameter is. This is the point were the 
Annotated Shader Language comes into play.

The Annotated Shader Language, or short ASL, allows it to add information like 
description and parameter range of a parameter to the program. Because all ASL 
elements are written in comments the shader program can still be compiled 
without using an ASL shader compiler.

Because developed ASL only as part of a rather small project there is still room 
for many enhancements. For example, at the moment it is only possible to 
annotate uniforms and whole files, but no per vertex parameters. We had no need 
for this function as we only used fragment shaders which do not have per vertex 
parameters.


\chapter{A First Look}
This chapter shows a simple example of an ASL annotated shader program to get 
a first impression what the language looks like:

\begin{lstlisting}
uniform sampler2D tex;

/**
 * Name: |Threshold|
 * Description: |This gives the threshold for coloring bright pixels. All pixels
 *     that have a color vector with a length greater than this value will be
 *     colored.|
 * Default: |1.33|
 * Range: |0, 1.44|
 */
uniform float threshold;

/**
 * Name: |Highlight Color|
 * Description: |Color used for highlighting the bright pixels.|
 * Default: |vec3(0, 0, 0)|
 * Range: |percent|
 * Control: |color, vector|
 */
uniform vec3 color;

/**
 * Name: |High-Key|
 * Description: |Highlights bright areas.|
 */
void main() {
    gl_FragColor = gl_Color
        * texture2D(tex, gl_TexCoord[0].xy);
    if (length(gl_FragColor.rgb) > threshold) {
        gl_FragColor.rgb = color;
    }
}
\end{lstlisting}
This shader colors all bright pixels by computing the vector length of the 
color's RGB vector. As you can see the uniforms \lstinline$threshold$ and 
\lstinline$color$ are annotated with additional information such as the name, 
a description, default value and allowed parameter range. This allows to the 
program using this shader to provide corresponding user interface elements.

The uniform \lstinline$tex$ is expected to be passed to all shaders by the 
program using the shader and is not considered as a parameter the user can 
directly change per shader. Therefore it has no annotation.

Not only the uniforms, but also the \lstinline$main()$ function is annotated 
with some general information about the shader.

\chapter{Annotation Comments}
Annotations are added to the shader program with special comments starting with 
two asterisks:
\begin{lstlisting}
/** <annotation comment> */
\end{lstlisting}
The comment has to occur before the element it is annotating.

Annotation comments can span multiple lines:
\begin{lstlisting}
/**
 * <line 1>
 * <line 2>
 */
\end{lstlisting}
All whitespace and asterisks before and after the line will be discarded. This 
allows also boxlike annotation comments:
\begin{lstlisting}
/**************************
 * <annotation comment>   *
 * <another line>         *
 *************************/
\end{lstlisting}

Within an annotation comment you define the individual annotations by 
a key-value pattern. The key is the first word in the line followed by a colon 
and than the value attributed to that key:
\begin{lstlisting}
/**
 * key: value
 */
\end{lstlisting}
The keys are case-sensitive.

Sometimes it is useful to split the value into multiple lines, e.g. when writing 
long descriptions. To do this you only have to indent the lines after the first.  
All lines with a higher indentation than the key will be treated as 
a continuation of the value.
\begin{lstlisting}
/**
 * key: with a value
 *     that spans more
 *     than one line.
 * another-key: value
 */
\end{lstlisting}
Tabs will be considered as 8 spaces in this context. However, you should not mix 
tabulators and spaces.

\section{General Shader Annotation}
To annotate the shader program itself write an annotation comment for the 
\lstinline$main()$ function. The allowed keys for annotations are described in 
the following subsections.

\key{Name}
Gives the name of the shader program. Should be short. This annotation is 
required in each ASL shader program.

\key{Description}
A description of the shader program.

\key{Include}
TODO

\subsection{Example}
\begin{lstlisting}
/**
 * Name: Example Shader
 * Description: This is just an example with a shader doing nothing.
 * Include: some-other-shader.fs
 *          and-another-shader.fs
 */
void main() {
    /* your code here */
}
\end{lstlisting}

\section{Uniform Annotation}
To annotate a uniform parameter of a shader program write an annotation comment 
before it.

The data type of the parameter should be taken from the parameter declaration by 
the ASL compiler. Our current implementation does not support samplers, but only 
scalars, vectors and matrices as data types for annotated parameters.

The allowed keys for annotations are described in the following subsections.

\key{Name}
Gives the name of the parameter. Should be short. This annotation exists because 
the variable names in programs are often not that clear without context. This 
annotations allows you to add a more clear name for the user interface. You also 
do not have to worry about characters not valid in variable names.

This annotation is required for each annotated parameter.

\key{Description}
A description what the annotated parameter does. Programs using ASL may present 
this description as tool tip or any other way that seem suitable to them.

\key{Default}
Gives the default value for the parameter. Specifying the value works the same 
as specifying a value in the standard shading language. The default value has to 
be of the exact same type as the parameter. The only exception are float 
parameters where specifying the default value as integer is also permissible.

\key{Range}
This gives the valid parameter range. The basic syntax is the minimum and 
maximum value separated by commas:
\begin{lstlisting}
Range: 0, 1
\end{lstlisting}
The allows the parameter to be between 0 and 1 whereby the bounds are included.

You can also specify the valid range for vectors and matrices. If you use the 
scalar form from above the range is applied to each element of the vector or 
matrix. To specify individual ranges for each element write one vector 
(respectively matrix) with the minimum values and one with the maximum values 
like this:
\begin{lstlisting}
Range: vec3(0, 0, -5), vec3(10, 10, 5)
\end{lstlisting}

If you want to use the smallest or highest possible value you do not have to 
type the numeric value. It is better to use the words \lstinline$min$ and 
\lstinline$max$:
\begin{lstlisting}
Range: min, max
\end{lstlisting}
This would be the equivalent to not specifying \lstinline$Range$ at all.

Instead of specifying the ranges manually there are some predefined ranges:
\begin{itemize}
    \item \verb$percent$ is the same as \lstinline$0, 1$
    \item \verb$byte$ is the same as \lstinline$0, 255$
    \item \verb$unsigned$ is the same as \lstinline$0, max$
    \item \verb$positive$ is the same as \lstinline$1, max$
\end{itemize}

Moreover, you can give a list of valid values within curly braces:
\begin{lstlisting}
Range: { 1, 2, 3, 5, 8, 13 }
\end{lstlisting}

\key{Control}
A vector may either represent a position or a color. In each case another 
control would be suitable in the user interface. With the \lstinline$Control$ 
key you can give the program using this shader a hint which control to use.

The value of \lstinline$Control$ key is a comma separated list. The program 
using the shader should use the first control it provides in the list. If the 
program provides none of the controls in the list it should fall back to its 
default control for the data type.

Each program may define it own controls it provides. However, there are some 
standard names for controls you should use; even though a program is not 
required to provide these.
\begin{itemize}
    \item \verb$default$ stands for the program's default control for the data 
        type of the parameter.
    \item \verb$color$ request the program to use a color picker as control.  
        Makes sense only for vectors.
    \item \verb$slider$ requests to use a slider as control. (In case of 
        a non-scalar this may be interpreted as per element.)
    \item \verb$spinbox$ requests to use a spin box as control. (In case of 
        a non-scalar this may be interpreted as per element.)
\end{itemize}

\subsection{Example}
\begin{lstlisting}
/**
 * Name: Highlight Color
 * Description: The color used to highlight the affected areas.
 * Default: vec3(1, 0, 0)
 * Range: percent
 * Control: color, default
 */
uniform vec3 highlightColor;
\end{lstlisting}
This is a possible annotation for a color value using three components (RGB).  
Each component is limited to the range from 0 to 1 and the program should use 
a color picker to control this parameter.

\chapter{Additional Features}

\section{MAIN Macro}

\appendix
\chapter{Limitations and Possible Future Enhancements}

\end{document}
