\documentclass{beamer}

%%%%%% Packages %%%%%
\usepackage[utf8]{inputenc}
\usepackage[T1]{fontenc}
\usepackage{ae,aecompl}
%\usepackage{scrpage2}
%\usepackage{amssymb}
%\usepackage{amsmath}
%\usepackage[flushmargin,ragged]{footmisc}
% \usepackage{graphicx}
%\usepackage{ifthen}
% \usepackage{lastpage}
\usepackage{ngerman}
\usepackage{enumerate}
% \usepackage{hhline}
\usepackage{listings}
%\usepackage{xr-hyper}
\usepackage{hyperref}
%\usepackage{units}
% \usepackage{xcolor}

%%%%% Eigene Format-Kommandos %%%%%
\newcommand{\makeframetitle}[1]{{\usebeamercolor[fg]{frametitle} \usebeamerfont{frametitle} #1}}

\newenvironment{questionFrame}[1]{\begin{frame} \center \vfill \makeframetitle{#1} \linebreak \linebreak \pause}{\vfill \end{frame}}

%%%%% Externe Dokumente %%%%%

%%%%% Silbentrennung %%%%%

%%%%% Weiteres %%%%%
\lstdefinelanguage[opengl]{Shader}[ANSI]{C}{morekeywords={in,layout,sampler2D,uniform,varying,vec3}}
\lstdefinelanguage[ASL]{Shader}[opengl]{Shader}{morekeywords={Name,Control,Description,Default,Depends,Range,ShaderName,ShaderDescription},moredelim=*[s][\itshape]{/**}{*/},moredelim=[is][\itshape]{|}{|}}
\lstset{basicstyle=\fontfamily{pcr}\selectfont\footnotesize,keywordstyle=\bfseries,language=[ASL]Shader}

\title{Shaderbasierte Bildbearbeitung}
\author{Jan Gosmann \and Denis John \and Markus Kastrop}
\date{12. Juli 2011}
\hypersetup{
  pdftitle={Shaderbasierte Bildbearbeitung},
  pdfauthor={Jan Gosmann, Denis John, Markus Kastrop},
  pdfkeywords={}
}

\begin{document}
\maketitle

%TODO: frame: Was ist das Ziel unseres Projekts?

\begin{questionFrame}{Wie macht man das modular und einfach erweiterbar?}
    Möglichkeit bieten, Shader-Sourcefiles direkt zu öffnen.
\end{questionFrame}

\begin{questionFrame}{Aber woher weiß das Programm, für welche Shadereinstellungen UI-Elemente benötigt werden?}
    {\Huge Annotated Shading Language}
\end{questionFrame}

\begin{frame}
    \frametitle{Was ist die Annotated Shading Lanuage (ASL)?}
    \begin{itemize}
        \item Erlaubt das Beschreiben der Shaderparameter mit "`Annotations"'.
            \begin{itemize}
                \item Name und Beschreibung
                \item Default-Wert
                \item Erlaubter Wertebereich
                \item Bevorzugte Controls (z.\,B. Color Picker oder Spin Box)
            \end{itemize}
        \item Dem Shader kann ein Name und eine Beschreibung gegeben werden.
    \end{itemize}
\end{frame}

\begin{frame}
    \centering \vfill
    \makeframetitle{One More Thing \dots}
    \vfill
\end{frame}

\begin{frame}
    \frametitle{Abhängigkeiten bzw. Dependencies}
    \begin{itemize}
        \item Ein Shader kann einen anderen laden und dessen Funktionen mitverwenden.
        \item ASL wird in den geladenen Abhängigkeiten ebenfalls ausgewertet.
        \item \lstinline$ASL_MAIN$-Makro erlaubt Unterscheidung, ob ein Shader als Abhängigkeit geladen wird.
        \item Unsere Implementierung verwendet einen Cache, so dass jede Abhängigkeit nur einmal im Speicher vorhanden ist.
    \end{itemize}
\end{frame}

\begin{frame}[fragile]
    \frametitle{Und so sieht es aus \dots }
    \framesubtitle{convolveMat3.asl.frag}
    \begin{lstlisting}
/**
 * ShaderName: ConvolveMat3
 * ShaderDescription: Convolves a matrix with the texture.
 */

uniform sampler2D tex;
uniform int texWidth;
uniform int texHeight;

vec4 getConvoluteValue(mat3 convMat) {
    /* Some complicated code. */
}

/* ... */
    \end{lstlisting}
\end{frame}
\begin{frame}[fragile]
    \frametitle{Und so sieht es aus \dots }
    \framesubtitle{convolveMat3.asl.frag (fortgesetzt)}
    \begin{lstlisting}
/* ... */

#ifdef ASL_MAIN
/**
 * Name: Convolution Matrix
 * Default: mat3(1, 2, 1, 2, 4, 2, 1, 2, 1)
 *
uniform mat3 convolutionMatrix;

void main() {
    gl_FragColor = getConvoluteValue(convolutionMatrix);
}
#endif
     \end{lstlisting}
\end{frame}
\begin{frame}[fragile]
    \frametitle{Und so sieht es aus \dots }
    \framesubtitle{gauss.asl.frag}
    \begin{lstlisting}
/**
 * ShaderName: Gaussfilter
 * ShaderDescription: Uses a gaussian filter with the given
 *     texture.
 * Depends: convolveMat3.asl.frag
 */

const mat3 gaussMat = mat3(1, 2, 1, 2, 4, 2, 1, 2, 1);

void main() {
    gl_FragColor = getConvoluteValue(gaussMat);
}
     \end{lstlisting}
\end{frame}

\begin{frame}
    \frametitle{Noch ein paar (technische) Infos}
    \begin{itemize}
        \item ASL ist dokumentiert.
        \item Gute Unit-Testabdeckung des Parsers mit 216 Test-Cases.
        \item Parser basiert auf Bison\footnote{http://www.gnu.org/software/bison/}
             und Flex\footnote{http://flex.sourceforge.net/}.
        \item Eigene Implementierung des GLSL-Präprozessors.
        \item Jeder ASL-Shader ist ein gültiger GLSL-Shader und umgekehrt.
    \end{itemize}
\end{frame}

\begin{frame}
    \frametitle{Und was fehlt?}
    \begin{itemize}
        \item Nur Unterstützung für Fragment-Shader.
        \item Nur Unterstützung für einfache Uniforms (d.h. kein
            \lstinline$varying$/\lstinline$in$, kein \lstinline$layout$-Qualifier,
            keine Structs oder Arrays).
        \item Keine arithmetischen Ausdrücke für Default-Wert und Wertebereich unterstützt.
        \item Keine Unterstützung für zusätzliche Texturen (Sampler).
        \item Cool wäre auch, wenn man direkt in ASL mehrere Renderdurchläufe definieren könnte.
        \item Weitere kleine Verbesserungen an verschiedenen Stellen denkbar.
    \end{itemize}
\end{frame}

\end{document}
